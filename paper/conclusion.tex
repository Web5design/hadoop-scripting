\section{Conclusion}

Pig and Jaql both succeed in the task of abstracting from the Map/Reduce pattern towards the actual data processing tasks.

Pig Latin is limited in expressiveness and a fallback to UDFs is often needed, whereas Jaql provides a
richer feature set being extendable with pure Jaql.

The time consuming part of developing in Pig is writing UDFs. If a library of commonly used UDFs is present (i.e. Pig's Open Source UDF Repository PiggyBank~\cite{pigPiggyBank} or company specific in-house librarys) a Pig Latin script is composed in very short time. For a lot of common tasks, native Pig Latin statements already are powerful enough and UDFs aren't needed.

Furthermore we would prefer Pig for ``easy'' querying tasks, since basic functionality is quickly accessible and well documented. But we think, that prototyping of more complex problems is faster done in Jaql than in Pig.

Since Pig and Jaql are still in development improvements may be seen, especially in performance, where
both aim on competing with native Java.

While Pig seems to be production-ready Jaql currently appears to be more of a research project. 

In conclusion, Pig and Jaql both ease programming for Hadoop by allowing rapid development, compared to
Hadoop's Java interface, but they are both in an early stage of development and cannot yet fully
compete with pure Java Hadoop performance in our macro benchmarks.