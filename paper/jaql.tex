\section{Jaql}

Jaql~\cite{jaqlWebsite} is a high-level scripting language for the JavaScript Object Notation (JSON).
It is able to run on Hadoop and break most requests down to Map/Reduce tasks.
Jaql heavily borrowes from SQL, XQuery, LISP, Pig Latin, JavaScript and Unix Pipes.~\cite{jaqlOverview}

Developed mainly inside IBM and with a quiet mailinglist Jaql currently faces
a serious lack of documentation and community. The documentation found online
is outdated and incomplete in most cases. In addition, Jaql was undergoing
major changes in the time of writing this paper.

Even though available, there is no need to write in a strict Map/Reduce pattern. At the point of execution (i.e. the end of a query statement) Jaql transforms the
parsed statement into another, equivalent but optimised statement. This step is
comparable to a query optimiser in modern database management systems. The optimised
query can in turn be transformed back to Jaql code, which is useful for debugging.

Jaql can be extended with user-defined functions, written either in Java or
in Jaql itself. However, it is not possible to use Jaql as a general-purpose programming
language, as it is not Turing-complete: It lacks both recursion and a universal loop
function.

The current Jaql implementation features three modes to run in: In stand-alone mode Hadoop
is not used at all and the jobs are not split in Map/Reduce tasks. When using Jaql
with Hadoop a so-called mini-cluster can be used, which is managed by Jaql and runs all tasks
on one computer in the same process (with one thread per Map/Reduce task). The last option is running Jaql on a traditional, external Hadoop cluster.