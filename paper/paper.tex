\documentclass{llncs}
\usepackage{makeidx}  % allows for indexgeneration
\usepackage[pdftex]{graphicx} % PNGs
\usepackage{amsmath, amssymb} % algebra
\usepackage[utf8x]{inputenc}
\usepackage[T1]{fontenc} 
\usepackage[procnames]{listings} % for sourcecode
\usepackage{graphviz} % graphs
\usepackage{array,multirow} % tables
\usepackage{afterpage} % figures
\usepackage{float} % figures

\restylefloat{figure}
\begin{document}
\pagestyle{headings}  % switches on printing of running heads
\mainmatter % start of the contributions
\title{Similarity Join on Hadoop}
\subtitle{An Evaluation of the Hadoop Map/Reduce Environment}
\titlerunning{Similarity Join}  % abbreviated title (for running head)
\author{Andrina Mascher, Tim Felgentreff}
\date{\today}
\authorrunning{Mascher, Felgentreff}   % abbreviated author list (for running head)
\tocauthor{Andrina Mascher, Tim Felgentreff (Hasso-Plattner-Institute)}
\institute{Map/Reduce Algorithms on Hadoop, Research Group Information Systems, Hasso-Plattner-Institut, Universität Potsdam, D-14482 Potsdam, Germany,\\
\email{\{andrina.mascher, tim.felgentreff\}@student.hpi.uni-potsdam.de}}

\maketitle
\begin{abstract}
  Bla Bla Bla yada yada yada
\end{abstract}
\section{Introduction} % 1 Seite
Lorem Ipsum
\section{Algorithm}
Some general thoughts
\subsection{Supporting Algorithms} % 0.6 Seiten
\paragraph{Jaccard}
\paragraph{Porter Stemmer}
\subsection{Phase 1}
\subsubsection{Map}
\subsubsection{Reduce}
\subsection{Phase 2}
\subsubsection{Map}
\subsubsection{Reduce}
\subsection{Phase 3}
\subsubsection{Map}
\subsubsection{Reduce} % 2-3 Seiten
\section{Benchmarking}
All of the aforementioned thoughts and architecture would be futile if it were not 
rewarded with an appropriate increase in performance or at least scalability. 
\begin{figure}[tb]
  \begin{center}
    \includegraphics[width=0.8\linewidth]{imgs/scaling}
  \end{center}
  \caption{Run-time development over growing input sizes}
  \label{fig:runtime}
\end{figure}
As we can gather from Fig. \ref{fig:runtime} the run-time complexity of the 
algorithm matches a graph in $\mathcal{O}(n^2)$, as we had shown theoretically.
This complexity class, however, means that we have not been able to reach a 
better complexity of the overall algorithm through map/reduce. For small data sets
running it in the cluster is even slower than running locally. Still, the 
gain comes with the massive parallelization of Hadoop, and thus is more and more 
apparent on large inputs.

The reasons we found for this are mainly in the way Hadoop distributes data 
between the map and reduce phases. Especially for small jobs with low per-node 
run-time the overhead of sending, in the worst-case scenario, all of our data 
over TCP/IP three times over outmatches any gain through 
parallel processing. This assertion appeared to hold when running on larger 
input-sets. It proved to be correct when we changed the algorithm from running 
four map/reduce jobs to only running three.

Running with one job less our running time went down by an almost constant fraction 
for each point of measurement, indicating that indeed a large amount of our input 
is send over the network in-between jobs.
%
\section{Lessons learned}
\begin{figure}[tb]
  \begin{center}
    %\includegraphics{reducers}
  \end{center}
  \caption{Processing Speed in Relation to Number of Reducers}
  \label{fig:reducers}
\end{figure}
Things about Hadoop \\
Crazy stuff
\section{Conclusions}
Whatever we thought, we can do, we do
\section*{Acknowledgements} % 2-3 Seiten
We thank Alexander Albrecht for his advice on the algorithm and his help in finding the optimal 
way at tackling the problem. We would also like to thank Professor Neumann for enabling us to 
attend this seminar.
%
\nocite{*}
\bibliographystyle{splncs}
\bibliography{hadoop}
\clearpage
\end{document}
