\section{Programming with Jaql} 

\begin{lstlisting}[language=jaql,caption=Wordcount in Jaql,float,label=jaqlwc]
read($src)
  -> expand splitArr($, " ")
  -> group by $w = ($) into [$w, count($)];
\end{lstlisting}

Jaql features two ways of passing parameters to a function. You can write the parameters
in brackets after the function name, as known from most other programming languages, i.e.\
\lstinline[language=jaql]!foo("bar")!. However, Jaql also features a new pipe syntax, more
consistent with the way build in Jaql statements are written. The value you ``pipe into''
a function will be past as the first argument, therefore \lstinline[language=jaql]!foo("bar", "baz")!
is equivalten to \lstinline[language=jaql]!"bar" -> foo("baz")!.

All values, either arguments or return values, are either JSON objects or Jaql functions.
Every function in Jaql returns a value. Jaql gives advanced functionallity for querying, transforming
and aggregating those values. An example can be seen in listing \ref{jaqlwc}. Note that
operations like {\tt expand} or {\tt group by} will loop through all element of the given array
(and potentially split this operation on multiple map/reduce tasks) you will have {\tt\$} referring
to the current element.

A statement is terminated by a semicolon and will at that point be optimized, as mentioned earlier.
However, instead of executing the statement, Jaql is able to return the optimized Jaql code by calling
{\tt explain} on the statement.