\section{Programming Markov Chain with Pig}                                              
                         
\begin{lstlisting}[language=pig,caption=Markov Chain in Pig Latin,label=pigmarkovstructure,columns=fullflexible]    
wikipedia = LOAD 'wikipedia' USING PigStorage('\t') AS (row:chararray);
tuples = FOREACH wikipedia GENERATE FLATTEN(splitsuc.SPLITSUC(*));
grouped = GROUP tuples by (word1, word2, word3, word4) PARALLEL 16;
grouped_counted = FOREACH grouped GENERATE group, COUNT(tuples);
STORE grouped_counted INTO 'wikipedia.sql' USING splitsuc.STORESQL();
\end{lstlisting}                  

For the implementation two UDFs had to be programmed. 
\begin{itemize}
\item The first one is {\tt SPLITSUC()}, which is called within a {\tt FOREACH} loop, processing each row of the input. It is splitting the rows into 4-word tuples with a tuple containing every word with its three predecessors.
\item The other one is {\tt STORESQL()}, that is transforming the 4-tuples into SQL Multi-INSERT statements ready to be imported into a database. The number of output files is determined by the number of reduce jobs.
\end{itemize}

These UDFs each are about 80 lines of code and consumed most of the development time needed for implementing the algorithm. Writing the Pig Latin code was a fairly easy task, since it only meant straight-forward implementing the abstract algorithm.



